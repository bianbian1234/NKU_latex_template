% !TEX encoding = UTF-8 Unicode
%% !TEX program = xelatex
%% !BIB program = biber
%% !TEX TS-program = xelatex
%% !BIB TS-program = biber
%%
%%  本模板方式编译: XeLaTeX + biber
%%
%%  注意: 在改变编译方式前应先删除 *.toc 和 *.aux 文件
%%
\documentclass[12pt,openright]{book}

% 引入NKThesis包
\usepackage[emptydoublepage]{NKThesis}   % 中文
%\usepackage[emptydoublepage,English]{NKThesis} % 英文

% 其它包按需添加
% \usepackage{amsmath}
% \usepackage{cases}
% \usepackage{multirow}
% 以下是后加的

 \usepackage{graphicx}
 \usepackage{epstopdf}% 加入eps图片用
 \usepackage{subfigure}
 \usepackage{amsmath}
 \usepackage{amssymb} %某些特殊数学符号
 \usepackage{bm}
 \usepackage{booktabs}
 \usepackage{float}
 \usepackage{threeparttable}  %给表格添加脚注用
 \usepackage{multicol} % 后添加的 分栏用 可以删去
 \usepackage{makecell}% 表格内换行用
 \usepackage{tabularx}%表格自动换行用
 \usepackage[figuresright]{rotating} %表格横置
 \usepackage{lscape} %页面横置(插入技术路线流程图用)
% \usepackage{geometry} %注意 不要打开 会引起前两页排版混乱
 \usepackage{pifont} %输入带圈字符

 \usepackage[bottom]{footmisc} % 脚注强行固定到页脚


\usepackage{listings}% 附录中加入代码用
\usepackage[framed,numbered,autolinebreaks,useliterate]{mcode}% 附录中加入代码用
\maxdeadcycles=1000

% % -----------------------------------------下面是暂存内容
% \usepackage{txfonts} % 设置公式字体为times new roman - 其中一种方案
% \let\mathbb=\varmathbb
% \DeclareSymbolFont{letters}{OML}{ztmcm}{m}{it}
% %上面两行用于修正公式中大写S后面空隙过大的问题


% \usepackage{times}% 设置公式字体为times new roman - 其中一种方案
% \usepackage{mathptmx}

% --------------------------------------------------
% % 对于换页导致的标题距离页眉过近,使用下面的方法处理
% \clearpage
% \ \ \vspace{-7mm}
% \section{研究背景和意义}\label{节:研究背景和意义}
% 或\subsection{研究背景和意义}\label{节:研究背景和意义}
% 或\subsubsection{研究背景和意义}\label{节:研究背景和意义}

% --------------------------------------------------
% 连续标题空行调整如下: 

% \vspace{\gapcheptersection}

%   \section{研究背景和意义}\label{节:研究背景和意义}

%   \vspace{\gapsectionsub}

%     \subsection{研究背景和研究目标}\label{小节:研究背景和研究目标} 

%     \vspace{\gapsubsectionsubsub}

%       \subsubsection{研究背景}


% --------------------------------------------------


%  加粗字体可用  {\jiacu 文字内容} 
% --------------------------------------------------

\usepackage[T1]{fontenc}   % 设置公式和正文英文字体为times - 当前方案
\usepackage{newtxtext, newtxmath} % 如果字体出现问题将这一行和前面一行打开
% \setmainfont{Times New Roman} %
\let\mathbb=\varmathbb
\DeclareSymbolFont{letters}{OML}{ztmcm}{m}{it}
%上面两行用于修正公式中大写S后面空隙过大的问题
% 如果英文和公式的字体出现问题,将NKTfonts.cfg中第11行打开。
% % https://liam.page/2017/01/10/Times-New-Roman-and-LaTeX/

\raggedbottom % 修正由于自动分页导致的段落间距拉长

\hypersetup{hidelinks} % 隐藏超链接方框

% \usepackage{etoolbox}
% \BeforeBeginEnvironment{tabular}{\zihaowu} %修改表格内字体为五号
%\usepackage{pdflscape}% 某些页面横排
% 修改图目录和表目录前面的标题字
\renewcommand{\listfigurename}{图目录}

\renewcommand{\listtablename}{表目录}


%\newcommand{\citeupp}[1]{\citeauthor{#1}\cite{#1}} % 用于替代citep的新命令

% 参考文献
\addbibresource{nkthesis.bib}

% 图片文件夹
\graphicspath{{image/}}

\includeonly{
	./tex/abstract,
	./tex/introduction,
	./tex/relatedwork,
	./tex/method,
	./tex/discussion,
	./tex/summary,
	./tex/references,
	./tex/acknowledgements,
	./tex/appendices,
	./tex/resume
}
\begin{document}

%%%%%%%%%%%%%%%%%%%%%%%%%%%%%%%%%%%%%%%%%%%%%%%%%%%%%%%%%%%%%%%%%%%%%%%%%%%%%%%%
%  设置基本信息
%  注意:  逗号`,'是项目分隔符. 如果某一项的值出现逗号, 应放在花括号内, 如 {,}
%%%%%%%%%%%%%%%%%%%%%%%%%%%%%%%%%%%%%%%%%%%%%%%%%%%%%%%%%%%%%%%%%%%%%%%%%%%%%%%%
\NKTsetup{
	% 封面设置
	论文题目(中文) = 中文题目,
	副标题         = ,
	论文题目(英文) = en title,
	论文作者       = 张三,
	学号           = 1111111111,
	指导教师       = 李四\ \  教授,
	申请学位       = 博士,
	培养单位       = 某学院,
	学科专业       = 某专业,
	研究方向       = 某方向,
	答辩委员会主席 = {待定},
	评阅人 = {匿名评审},
	中图分类号     = ,
	UDC            = ,
	学校代码       = 10055,
	论文完成时间   = 二〇二一年五月,
	% 保密设置
	密级           = 公开,	% 公开 | 限制 | 秘密 | 机密, 若为公开, 不填以下三项
	非公开论文编号 = ,
	保密期限       = ,
	审批表编号     = ,
	% 其他信息
	批准日期       = ,
	答辩日期       = ,
	论文类别       = 博士, % 博士 | 学历硕士 | 专业学位硕士 | 同等学力硕士
	院/系/所       = 某学院,
	联系电话       = 11111111111,
	Email          = aa@aa.com,
	通讯地址(邮编) = 300071,
	备注           = {}
}

%%%%%%%%%%%%%%%%%%%%%%%%%%%%
% 论文开始部分
%%%%%%%%%%%%%%%%%%%%%%%%%%%%
% 摘要
% !TeX root = ../main.tex
% -*- coding: utf-8 -*-


\begin{zhaiyao}

中文摘要内容







\end{zhaiyao}




\begin{guanjianci}
关键词内容
\end{guanjianci}



\begin{abstract}
英文摘要内容

\end{abstract}



\begin{keywords}
英文关键词
\end{keywords}
% 论文目录
\tableofcontents
% 列出图表目录,如果需要可取消注释-在图表目录中序号之前加入“图”字和“表”字
{%
\let\oldnumberline\numberline%
\renewcommand{\numberline}{\figurename~\oldnumberline}%
\listoffigures%
}
{%
\let\oldnumberline\numberline%
\renewcommand{\numberline}{\tablename~\oldnumberline}%
\listoftables%
}

%%%%%%%%%%%%%%%%%%%%%%%%%%%%
% 论文主体章节
%%%%%%%%%%%%%%%%%%%%%%%%%%%%

% \clearpage
% \thispagestyle{empty}



% !TeX root = ../main.tex
% -*- coding: utf-8 -*-
% !TeX root = ../main.tex
% -*- coding: utf-8 -*-

\chapter{绪论}
\label{章:绪论}

引用使用textcite和parencite
如:\textcite{RN1327}认为,。。。。。

和这个理论\parencite{RN275}.

% !TeX root = ../main.tex
% -*- coding: utf-8 -*-


\chapter{文献综述} 
\label{章:文献综述}

引用使用textcite和parencite.


% !TeX root = ../main.tex
% -*- coding: utf-8 -*-

\chapter{模型及方法}


% !TeX root = ../main.tex
% -*- coding: utf-8 -*-

\chapter{实证研究}\label{章:实证研究}
    















% !TeX root = ../main.tex
% -*- coding: utf-8 -*-
\chapter{结论}


%%%%%%%%%%%%%%%%%%%%%%%%%%%%
% 论文其他信息
%%%%%%%%%%%%%%%%%%%%%%%%%%%%
% !TeX root = ../main.tex
% -*- coding: utf-8 -*-

\chapter*{附录}


\section*{附录1:xxxx}

% !TeX root = ../main.tex
% -*- coding: utf-8 -*-


\printbibliography[title=参考文献]
% \addcontentsline{toc}{chapter}{参考文献} %向目录中添加条目,以章的形式
% !TeX root = ../main.tex
% -*- coding: utf-8 -*-

%\makeschapterhead{致谢}
\chapter*{致谢}

    {\fs \baselineskip 16pt \fontsize{12}{6} 致谢内容。致谢内容。致谢内容。致谢内容。致谢内容。致谢内容。致谢内容。致谢内容。致谢内容。致谢内容。致谢内容。致谢内容。致谢内容。致谢内容。致谢内容。致谢内容。致谢内容。致谢内容。致谢内容。致谢内容。致谢内容。致谢内容。致谢内容。致谢内容。致谢内容。致谢内容。致谢内容。致谢内容。致谢内容。致谢内容。致谢内容。

    致谢内容。致谢内容。致谢内容。致谢内容。致谢内容。致谢内容。致谢内容。致谢内容。致谢内容。致谢内容。致谢内容。致谢内容。致谢内容。致谢内容。致谢内容。致谢内容。致谢内容。致谢内容。致谢内容。致谢内容。致谢内容。致谢内容。

    致谢内容。致谢内容。致谢内容。致谢内容。致谢内容。致谢内容。致谢内容。致谢内容。致谢内容。}





% !TeX root = ../main.tex
% -*- coding: utf-8 -*-

\chapter*{个人简历}

{\baselineskip 16pt \fontsize{10.5}{10}  个人简历内容}

\section*{\leftline{研究生期间发表论文:}}

{ \baselineskip 16pt \fontsize{10.5}{16}  个人简历内容}

% 学术论文研究成果按发表的时间顺序列出
% (已发表的列在前面,已接收待发表的放在后面)
% 格式方便阅读为主可参考百度学术Google学术







\end{document}
